\documentclass[UTF8,11pt,a4paper]{ctexart}

\usepackage[left=2.2cm,right=2.2cm,top=2.2cm,bottom=2.4cm]{geometry}
\usepackage{amsmath,amssymb,mathtools}
\usepackage{booktabs}
\usepackage{longtable}
\usepackage{array}
\usepackage{graphicx}
\usepackage{subcaption}
\usepackage{float}
\usepackage{siunitx}
\usepackage{xcolor}
\usepackage{hyperref}
\usepackage{enumitem}
\usepackage{listings}
\usepackage{fancyvrb}

\hypersetup{
  colorlinks=true,
  linkcolor=blue!60!black,
  urlcolor=teal!60!black,
  citecolor=blue!60!black
}

\setlist[itemize]{leftmargin=1.8em}
\setlist[enumerate]{leftmargin=1.8em}

\lstdefinestyle{pythonstyle}{
  language=Python,
  basicstyle=\ttfamily\small,
  keywordstyle=\color{blue!70!black},
  commentstyle=\color{green!40!black},
  stringstyle=\color{orange!60!black},
  numbers=left,
  numberstyle=\tiny\color{gray},
  stepnumber=1,
  numbersep=8pt,
  showstringspaces=false,
  breaklines=true,
  frame=single,
  framerule=0.3pt,
  rulecolor=\color{black!20},
  tabsize=2,
  captionpos=b
}

\title{CS336 Assignment 1 Writeup}
\author{ha0xin}
\date{2026-02-12}

\begin{document}
\maketitle
\tableofcontents
\newpage

\section{Unicode、BPE 与 Tokenizer 题目}

\subsection{Problem \texttt{unicode1} 与 \texttt{unicode2}}

\begin{enumerate}
  \item \texttt{chr(0)} 返回 ASCII 的 NUL(\texttt{U+0000})。
  \item \texttt{repr(chr(0))} 显示其转义形式(如 \texttt{"\textbackslash x00"}),而 \texttt{print(chr(0))} 不可见。
  \item NUL 在字符串中通常作为不可见控制字符存在,很多显示与处理链路会忽略或特殊处理。
  \item UTF-8 相比 UTF-16/UTF-32 在网页文本上更紧凑、生态兼容最好,适合字节级 tokenizer。
  \item 错误 UTF-8 解码函数逐字节解码会破坏多字节字符,例如 \verb|b'\xe7\x89\x9b'|(“牛”)会失败或错误。
  \item 非法两字节序列示例:\verb|b'\xc0\xaf'|
\end{enumerate}

\subsection{BPE 训练结果与分析 (Problem \texttt{train\_bpe\_*})}

\subsubsection{TinyStories 10k}

训练耗时:\SI{36.58}{s},峰值内存:\SI{254.65}{MB}, 最长 token:\texttt{' accomplishment'}(15 bytes,含前导空格)

该最长 token 语义自然,BPE 在该语域下学到的是高频可解释子词。

\subsubsection{OWT 32k}

训练耗时:\SI{836.67}{s},峰值内存:\SI{17777.30}{MB},最长 token 长度 64 bytes,内容为:\\
\texttt{ÃÂÃÂÃÂÃÂÃÂÃÂÃÂÃÂÃÂÃÂÃÂÃÂÃÂÃÂÃÂÃÂ}

这是网页语料的噪声(可能是字符串解码错误)造成的。

\subsubsection{训练瓶颈说明}

通过Profile, BPE 训练主耗时来自预分词与 pair 计数维护。分别对应 \texttt{\_pretokenize\_parallel} 和 \texttt{pair\_counts}/\texttt{pair\_occurrences} 与 \texttt{\_update\_pair}。

\subsection{Tokenizer 实验(Problem \texttt{tokenizer\_experiments})}

\begin{itemize}
  \item 将 TinyStories tokenizer 用于 TinyStories 的压缩率:\SI{4.248}{bytes/token}
  \item 将 OWT tokenizer 用于 OWT 的压缩率:\SI{4.682}{bytes/token}
  \item OWT 样本用 TinyStories tokenizer:\SI{3.261}{bytes/token},我们发现压缩率相比用自身 tokenizer 降低了。
  \item OWT valid 编码吞吐: \SI{7.25}{MiB/s}
  \item TinyStories valid 编码吞吐:\SI{10.90}{MiB/s}
\end{itemize}

Pile 耗时估算:
\[
  t \approx \frac{825\times 1024^3}{7.6011956\times 10^6} \approx 1.165\times 10^5\,\text{s}
  \approx 1.35\,\text{days}.
\]

为什么用 \texttt{uint16}:词表规模最大 32k,token id 全在 \([0,65535]\) 内,\texttt{uint16} 足够且高效。

% \section{实现代码展示(关键片段)}

% \subsection{训练参数入口与 FFN 选择}

% \lstinputlisting[
%   style=pythonstyle,
%   caption={\texttt{scripts/train\_lm.py}: 命令行参数(\texttt{--d-ff}, \texttt{--ffn-type})与模型构建},
%   firstline=350,
%   lastline=447
% ]{scripts/train_lm.py}

% \subsection{SwiGLU / SiLU 分支实现}

% \lstinputlisting[
%   style=pythonstyle,
%   caption={\texttt{scripts/train\_lm.py}: \texttt{ffn\_type} 分支(SwiGLU vs SiLU)},
%   firstline=141,
%   lastline=165
% ]{scripts/train_lm.py}

% \subsection{实验汇总数据重建脚本}

% \lstinputlisting[
%   style=pythonstyle,
%   caption={\texttt{scripts/refresh\_assignment1\_reports.py}: 发散/批量/消融汇总},
%   firstline=113,
%   lastline=207
% ]{scripts/refresh_assignment1_reports.py}


\section{计算题详细推导}

\subsection{Problem \texttt{transformer\_accounting}}

\subsubsection{参数量推导(GPT-2 XL-shaped)}

记:词表大小 \(V\)、层数 \(L\)、隐藏维度 \(d\)、FFN 维度 \(d_{ff}\)。
本实现采用 SwiGLU(三矩阵),且 \texttt{embedding} 与 \texttt{lm\_head} 不绑权。

\paragraph{参数分解}
\begin{align}
P_{emb} &= Vd,\\
P_{head} &= Vd,\\
P_{attn/layer} &= 4d^2\quad(Q,K,V,O),\\
P_{ffn/layer} &= 3dd_{ff}\quad(W_1,W_2,W_3),\\
P_{norm/layer} &= 2d,\\
P_{norm,final} &= d.
\end{align}

总参数:
\[
P = 2Vd + L(4d^2 + 3dd_{ff} + 2d) + d.
\]

代入 \(V=50257, L=48, d=1600, d_{ff}=6400\):
\[
P = 2{,}127{,}057{,}600.
\]

float32 参数内存:
\[
M_{param} = 4P \approx 7.92\,\text{GiB}.
\]

\subsubsection{前向 FLOPs 推导}

设 batch 为 \(B\),上下文长度为 \(T\)。使用矩阵乘规则:
\(A\in\mathbb{R}^{m\times n},B\in\mathbb{R}^{n\times p}\Rightarrow 2mnp\) FLOPs。

\paragraph{单层 FLOPs}
\begin{align}
F_{QKV} &= 3\cdot 2BTd^2 = 6BTd^2,\\
F_{attn\,mat} &= 2BT^2d\,(QK^\top) + 2BT^2d\,(AV)=4BT^2d,\\
F_{out\,proj} &= 2BTd^2,\\
F_{FFN} &= 2BTdd_{ff} + 2BTdd_{ff} + 2BTd_{ff}d = 6BTdd_{ff}.
\end{align}

最终输出头:
\[
F_{head}=2BTdV.
\]

总 FLOPs:
\[
F = L(8BTd^2 + 4BT^2d + 6BTdd_{ff}) + 2BTdV.
\]

代入 GPT-2 XL-shaped, \(B=1,T=1024\):
\[
F\approx 4.513\times 10^{12}.
\]

\subsubsection{FLOPs 占比与规模变化}

\begin{table}[H]
\centering
\caption{固定 \(T=1024,B=1\) 时,不同模型 FLOPs 组成占比}
\begin{tabular}{lrrrr}
\toprule
模型 & Attention投影 & Attention矩阵乘 & FFN & LM Head \\
\midrule
gpt2-small  & 16.58\% & 11.06\% & 49.75\% & 22.61\% \\
gpt2-medium & 19.96\% & 9.98\%  & 59.87\% & 10.20\% \\
gpt2-large  & 21.40\% & 8.56\%  & 64.20\% & 5.84\%  \\
gpt2-xl     & 22.30\% & 7.14\%  & 66.91\% & 3.65\%  \\
\bottomrule
\end{tabular}
\end{table}

结论:模型越大(固定 \(T\)),FFN 与投影占比上升,\(T^2\) 注意力和 LM Head 占比下降。

\subsubsection{上下文扩展到 T=16384}

GPT-2 XL-shaped 时:
\[
F_{16384} \approx 1.495\times 10^{14},\quad
\frac{F_{16384}}{F_{1024}}\approx 33.13.
\]

由于 \(T^2\) 项增长,Attention 矩阵乘占比升至约 55.15\%,成为主导项。

\subsection{AdamW Accounting(按题面简化口径)}

题面口径使用 \(d_{ff}=4d\) 且 FFN 记作 \(W_1\)-SiLU-\(W_2\)。

\subsubsection{内存分解}

\paragraph{参数项}
\[
P = 2Vd + L(12d^2 + 2d) + d.
\]

\[
M_{param}=4P,\quad M_{grad}=4P,\quad M_{opt}=8P.
\]

\paragraph{激活项(上界近似)}
按题面列出的组件进行近似汇总:
\[
A \approx L(16BTd + 2BhT^2) + BTd + 2BTV,
\]
\[
M_{act}\approx 4A.
\]

故峰值近似:
\[
M_{total}\approx M_{param}+M_{grad}+M_{opt}+M_{act}=16P+4A.
\]

\subsubsection{代入 GPT-2 XL-shaped}

取 \(V=50257,T=1024,L=48,d=1600,h=25\),得到
\[
M_{total}(B)\approx 15{,}517{,}753{,}344\cdot B + 26{,}168{,}601{,}600\quad (\text{bytes}),
\]
即
\[
M_{total}(B)\approx 14.452B + 24.371\quad (\text{GiB}).
\]

\(
B_{max}=\left\lfloor\frac{80-24.371}{14.452}\right\rfloor=3
\),所以 80GiB 下最大 batch 约 3。

\subsubsection{AdamW 单步 FLOPs 与训练时长}

单步 FLOPs 近似:
\[
F_{step}\approx 3F_{forward}+cP
\]
其中 \(c\) 为常数(参数更新项),主导仍是前向+反向。

按 A100 float32 峰值 \(19.5\,\text{TFLOP/s}\)、MFU=50\%、400k steps、batch=1024:
\[
\text{days}\approx 5115.
\]
数量级结论:单卡 float32 下该规模训练不现实。

\section{训练实验与图表}

\subsection{学习率扫描与发散证据(TinyStories)}

\begin{table}[H]
\centering
\caption{Learning Rate Sweep(TinyStories)}
\begin{tabular}{rrl}
\toprule
lr & best val loss & 备注 \\
\midrule
1e-4 & 2.1700 & 稳定收敛 \\
3e-4 & 1.7620 & 稳定收敛 \\
1e-3 & 1.5593 & 最优区间 \\
3e-3 & 1.6799 & 开始退化 \\
1e-2 & 2.5253 & 明显退化 \\
3e-2 & 3.7422 & 高损失不稳定 \\
1e-1 & 4.4049 & 高损失不稳定 \\
3e-1 & 4.2369 & 出现尖峰(max val=33.63) \\
1e0  & 9.2687 & 发散(max val=149.86) \\
3e0  & 9.2687 & 发散(max val=876.86) \\
\bottomrule
\end{tabular}
\end{table}

主实验(\texttt{ts\_main\_lr1e3\_20k})达到
\[
\text{best val loss}=1.3730\ (<1.45).
\]

\begin{figure}[H]
  \centering
  \includegraphics[width=0.78\textwidth]{artifacts/figures/tinystories_lr_sweep_val.png}
  \caption{TinyStories 学习率扫描(验证集)}
\end{figure}

\begin{figure}[H]
  \centering
  \includegraphics[width=0.78\textwidth]{artifacts/figures/tinystories_lr_sweep_train.png}
  \caption{TinyStories 学习率扫描(训练集)}
\end{figure}

\begin{figure}[H]
  \centering
  \includegraphics[width=0.78\textwidth]{artifacts/figures/tinystories_lr_divergence_val.png}
  \caption{高学习率发散探测(验证集)}
\end{figure}

\noindent 发散判据采用验证损失相对初始值的显著放大。补跑结果如下:
\begin{itemize}
  \item \texttt{lr=1.0}: \(\,149.86 / 9.27 \approx 16.17\,\)
  \item \texttt{lr=3.0}: \(\,876.86 / 9.27 \approx 94.61\,\)
\end{itemize}
对应日志:\texttt{slurm\_logs/train\_lm\_1478.out}、\texttt{slurm\_logs/train\_lm\_1479.out}。

\subsection{Batch Size 实验}

\begin{table}[H]
\centering
\caption{batch size 从 1 到显存上限探测}
\begin{tabular}{rrlrrl}
\toprule
batch & lr & status & best val loss & tokens seen & 备注 \\
\midrule
1   & 3e-4   & ok  & 2.9371 & 1.28M  & 可训练 \\
32  & 1e-3   & ok  & 1.6665 & 40.96M & 可训练 \\
64  & 1e-3   & ok  & 1.5593 & 81.92M & 可训练 \\
128 & 1.5e-3 & ok  & 1.4614 & 163.84M & 可训练 \\
256 & 1.5e-3 & ok  & 2.2109 & 19.66M & 显存仍可容纳(300 iter 探测) \\
512 & 1.5e-3 & oom & - & - & 首个 forward 即 OOM \\
\bottomrule
\end{tabular}
\end{table}

\begin{figure}[H]
  \centering
  \includegraphics[width=0.78\textwidth]{artifacts/figures/tinystories_batch_size_val.png}
  \caption{Batch size 对比(同训练步数)}
\end{figure}

\begin{figure}[H]
  \centering
  \includegraphics[width=0.78\textwidth]{artifacts/figures/tinystories_batch_size_token_matched_val.png}
  \caption{Batch size 对比(token-matched)}
\end{figure}

\noindent 因此在当前 32GB 显存与该模型配置下,最大可运行 batch 位于 \(256 < B_{\max} < 512\)。
OOM 证据见 \texttt{slurm\_logs/train\_lm\_1475.err}。

\subsection{消融实验}

\begin{table}[H]
\centering
\caption{TinyStories 消融(5k steps)}
\begin{tabular}{lr}
\toprule
设置 & best val loss \\
\midrule
baseline(pre-norm + RMSNorm + RoPE + SwiGLU) & 1.5593 \\
去 RMSNorm(lr=1e-3) & 1.5773 \\
去 RMSNorm(lr=3e-4) & 1.7679 \\
post-norm & 1.5553 \\
NoPE(去 RoPE) & 1.6537 \\
SiLU(替换 SwiGLU,\texttt{d\_ff=2048}) & 1.5835 \\
\bottomrule
\end{tabular}
\end{table}

\begin{figure}[H]
  \centering
  \includegraphics[width=0.78\textwidth]{artifacts/figures/tinystories_ablations_val.png}
  \caption{TinyStories 各消融验证曲线}
\end{figure}

\begin{figure}[H]
  \centering
  \includegraphics[width=0.78\textwidth]{artifacts/figures/tinystories_rmsnorm_ablation_val.png}
  \caption{RMSNorm 消融细节曲线}
\end{figure}

\noindent 本次 SiLU 消融按 handout 要求做参数量匹配:\(
d_{ff}=4\times d_{model}=4\times512=2048
\)。

\subsection{OpenWebText 主实验}

\begin{itemize}
  \item TinyStories 主实验 best val loss:1.3730
  \item OWT 主实验 best val loss:4.0370
\end{itemize}

\begin{figure}[H]
  \centering
  \includegraphics[width=0.78\textwidth]{artifacts/figures/ts_vs_owt_main_val.png}
  \caption{TinyStories 与 OWT 主实验验证曲线对比}
\end{figure}

解释:OWT 数据分布更复杂、熵更高,在同模型规模与 token 预算下拟合不足,因此 loss 更高,生成质量更弱。

\section{生成样例与质量分析}

\subsection{TinyStories 生成(满足 EOT 终止)}

\lstinputlisting[
  style=pythonstyle,
  caption={TinyStories 生成样例(满足 EOT 终止)},
  firstline=1,
  lastline=10
]{artifacts/experiments/lm/ts_main_lr1e3_20k/generated.txt}
该样例编码长度 169 token,并在首个 \texttt{<|endoftext|>} 终止,满足作业要求。

\subsection{OWT 生成(满足至少 256 token)}

\lstinputlisting[
  style=pythonstyle,
  caption={OWT 生成样例},
  firstline=1,
  lastline=10
]{artifacts/experiments/lm/owt_main_lr1e3_20k/generated.txt}

该样例编码长度 262 token,满足“至少 256 token”要求。

\subsection{影响生成质量的因素}

\begin{enumerate}
  \item 数据复杂度:OWT 的主题与词汇更复杂,难度显著高于 TinyStories。
  \item 模型容量与训练预算:17M 级模型在固定 token 预算下对 OWT 表达能力不足。
  \item 解码超参:temperature/top-k/top-p 会显著改变流畅度与多样性。
\end{enumerate}

\end{document}
